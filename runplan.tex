\section{Mu2e Commissioning and Data Taking Plan}
\label{sec:runplan}
Data taking phase descriptions and goals
\subsection{Un-integrated test stand running and Offline computing needs}
\subsection{Cosmic ray commissioning run}

\subsection{Early Beam Configuration}
Low intensity, unstable conditions.



\subsection{ Normal Stable Running Configuration}

The main goal of Mu2e is to detect conversion electrons from muons that stop in the stopping target.  The overall design of the physical apparatus is optimized for this goal.  The normal run configuration will optimize the ability of the experiment to produce and record potential conversion electron events, assuming stable and optimized beams.  A rough description of that configuration is:
\begin{itemize}
  \item Proton beam provided as extracted pulses with good extinction between pulses and reasonable intensity uniformity between pulses (SDF < X)
  \item Proton beam extracted at nominal intensity, focused as tightly as possible on the production target
  \item Detector solenoids at nominal field strength
  \item TS3 collimator set to select a mu- beam
  \item DAQ system configured for data taking in the window 450 -> 1705 ns relative to the nominal proton bunch arrival time at the production target
  \item Onspill Trigger configured for optimal efficiency for recording high momentum (p>80 MeV/c) electrons and positrons.
  \item Remaining Onspill trigger and data handling bandwidth allocated to dedicated calibration channels
  \item Offspill trigger configured to record cosmic rays useful for background estimation and detector calibration and alignment
\end{itemize}

\subsection{Special Runs}
There are several special running configurations that Mu2e will use to enhance non-conversion electron signals that are inaccessible or greatly attenuated during normal stable running configuration, but which are valuable for estimating physics backgrounds to conversion electrons.  Special run configurations will also be necessary when them beam conditions are non-optimal.  Some non-beam calibration run configurations will also be used.  These are described below.

unstable beam minimal feedback configuration.

RMC and RPC configuration
pi e nu configuration
mu Michel edge configuration

\subsection{Calibration Run Configurations}
calo source calibration config.

\subsection{Run 1}
Goals and parameters of run 1.
\subsection{Run 2}
Goals and parameters of run 2.
