\section{Simluations (Dave + Di Falco)}
\label{sec:simulation}
Describe the basic problem of Mu2e simulation: rare events, very
Describe the G4 model
Describe the detector response simulations (refer to existing docs for details).
Describe the physics generators

Brief history of existing simulation campaigns.  Show some plots.
Describe the overall plan going forwards: beam pileup from data, signals modeled from simulated tops.

Describe the test stand and extracted position simulation configurations and strategies.



\subsection{Mock Data}


Mock Data is developed using the output of large-scale production. We combine conversion signals (of a given conversion branching rate) with decay-in-orbit (DIO) tail backgrounds (reaching below the trigger threshold to 75 MeV/c), cosmic backgrounds (from either the CRY or CORSIKA generator), radiative pion and muon capture backgrounds and pile-up. These are then passed through our digitization and reconstruction framework to create ``data-like" samples.

The use of these samples is two-fold: 1) they can be used to develop analysis tools and strategies and 2) they can help us understand our data-production needs.

Mock data is currently built using custom Mu2e scripts. These scripts build .fcl files that utilize the art ``SamplingInput" functionality. This allows the weighted combination of a number of input .art files.


\textcolor{red}{placeholder, text will go into more details (Sophie)}

%\textcolor{red}{should ensembles be discussed here? (Sophie)}