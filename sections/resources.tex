\section{Estimated Computing Resource Needs}
\label{sec:resources}
Give a history of previous resource estimates and the timeline.
Give the current underpinning numbers for the resource estimates.
Describe the manpower resource estimates, both CSAID and collaboration.
Describe computing resources available across the collaboration (FNAL, INFN, NERSC, EAF, ...) and our plan for using those.


\subsection{Computing resources}

%\red{Moved from introduction, adapt as needed}

%Mu2e offline computing will negotiate a Technical Statement of Work (TSW) with the Fermilab  Computing Sector (CS) to ensure that Mu2e has access to the computing resources required to meet its needs. For example, offline computing has implemented a conditions database with support for intervals of validity (IOV) closed at both ends, non-overlapping chains of IOVs and a hierarchical tagging system. The database proper is hosted by SCD who also provide a caching web based front end that supports web based readonly access for many clients at the same time. A fraction of these resources will be supported by Mu2e (e.g. new tape media and the imputed cost to SCD for Mu2e’s use of CPU, disk, tape library, and database resources), while others will be provided at no cost (e.g. the services provided by FIFE). The TSW with CS will also include access to services supported by the Fermilab Core Computing Division (CCD), including collaboration and communication tools.

%In addition to the capabilities available at FNAL, Mu2e will seek external resources to execute computing intensive tasks. For example, large scale Monte Carlo simulations could be performed using opportunistic cycles on the Open Science Grid (OSG) or by requesting time at DOE High Performance Computing (HPC) centers. Several Mu2e collaborators can ask for cycles on HPC resources available at their institutions (LBNL, ANL,...). International contributions are also expected to be provided for specific tasks, such as processing calibration data or reconstructing simulated events. 
