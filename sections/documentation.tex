
\section{Documentation and training}
\label{sec:documentation}

The Mu2e collaborations includes more than 200 members, with new collaborators regularly joining the experiment. To ensure the long term success of the experiment, a harmonized documentation effort, coupled with consistent training for both newcomers and existing users, is paramount. The documentation related to the computing aspects is accessible on a variety of platforms, each with specific goals and access policies.

\subsection{Documentation}

The existing Mu2e wiki provides the starting point for the computing documentation. The wiki pages are divided into a public side containing general information about Mu2e computing, and a private side storing sensitive information. Users need a FNAL Single Sign-On service domain account to access the private side. The public Mu2e wiki is divided into three section: the Mu2e experiment, Mu2e computing, and practical information. The computing section includes six topics:
\begin{itemize}
\item {Computing tutorial :} provides an overview of the computing, guided tutorials, and introductory references
\item {Accounts, Authentication, and Infrastructure :} describes the various types of account and authentication procedures and the general computing infrastructure
\item {Code :} contains information about the Offline code base, the development tools, the coding standards, the fcl language, the Mu2e data products, and various utilities  
\item {Grids, Workflows, and Data Handling;} reviews procedures to prepare, submit and manage large jobs, upload files, dCache, tape,....
\item {Getting help :} lists the various resources to get help and additional information
\item {Management :} includes information about the FNAL and Mu2e computing management structure and related tasks
\end{itemize}
The wiki is maintained by the collaboration at large, and each collaborator is requested to document their work. Mu2e plans to conduct an annual "wiki review" to systematically identify outdated material, add missing documentation, and review actual content. 

The Mu2e software repositories are hosted on the GitHub platform, which offers a ticketing system to facilitate debugging, revisions and updates from the community of users. This features has been extensively used by the collaboration, and continue to be the main system to track code development tasks. Version-dependent documentation is also hosted on GitHub.




\subsection{Training}

Mu2e training is primarily aimed at new users, particularly younger early career users (such as new graduate students and post-docs). The primary training for new Mu2e members is done twice a year during dedicated sessions, usually a day workshop following a collaboration meeting, to ensure both new collaborators and experts are present in person. A zoom component is also offered for those who cannot travel. Training material is maintained throughout the year to provide "at home" learning as well. The training follows a hands-on example-based task format, starting with basic examples and moving gradually towards more advanced material. 

The tutorial material is divided into three sections: General Offline Mu2e computing, TrkAna/analysis, and event visualization. Within the GitHub repositories of the TrkAna and REveMu2e packages are a set of tutorials with gradually increasing difficulties. TrkAna, tutorials exist for both Python and ROOT C++-based analyses. The basic tutorial shows users how to import the information to their analysis platform, manipulate it, add selection cuts, and make basic histograms. The more advanced tutorials provide users with a real analysis problem and demonstrate the method to perform maximum likelihood fits to a Mock Data sample to provide point estimates of the signal and background yields. The REveMu2e tutorial starts by showing the user how to visualize tracks, hits, and clusters in the three main detectors and demonstrates the interactive features and event navigation. Training for more general Mu2e offline software development is also offered to users wishing to contribute to the source code. This tutorial includes setting up and building the Mu2e software environment and running a few simple tasks to modify the C++ source code. As the software evolves rapidly, the training material is updated frequently to ensure it remains always functional.