\section{Communication Tools}
\label{sec:communication}

Lines of communication need to flow among developers, among analyzers, and between developers and analyzers. Mu2e uses the following tools to communicate along all lines: email lists, Slack, doc-db, and GitHub.

Slack is the major communication mode used for quick-response and day-to-day communication such as general announcements, questions, and debugging of active issues. It is also to communicate with users to help debug. Mu2e has a premium plan such that all conversations are backed up and stored and Slack can be queried to allow users to look up previous conversations in all public channels. It is also used by developers to announce important hardware/software updates to analyzers that will affect their workflows. Communication for Analysis Reviews and internal Analysis Group discussions was recently migrated to dedicated slack channels, replacing the previous use of hypernews for this purpose.

Email lists are used mostly to announce meetings and important events of broad general interest to the entire collaboration. The most important email lists are linked directly to Slack channels to allow maximum reach to the intended audience. Any major software updates or infrastructure changes are also announced to relevant listservs.

Doc-db is used to store meeting agendas, presentations, and reports. Private groups can be created within doc-db, such that only members of a given Analysis Group can see these documents. This allows for independent analyses to be conducted privately while retaining all relevant documents in a common repository. 

Github is used to communicate between software developers. Analyzers can post technical issues and feature requests. Developers review pull requests. Issues can be searched to find open projects in search of new personnel. A system for labeling issues consistently across the Mu2e repositories is under development.

