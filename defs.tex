% Mark any meta-comments, eg; \fixme{This section needs work.}
%  4/1/14 art-workbook.cls now defines this \newcommand{\fixme}[1]{\textbf{FIXME:} \textit{#1}}
%The following hides all fixmes in PDF (gotta turn off the \newcommand{\fixme} above)
%\newcommand{\fixme}[1]{}

%\newcommand{\future}[1]{\textbf{FUTURE:} \textit{#1}}
%The following hides all futures (leaves text) in PDF (gotta turn off the \newcommand{\future} above)
\newcommand{\future}[1]{ \textit{#1}}

% The following command either just inserts or throws out its argument, depending on 
% which definition is uncommented.
%
%\newcommand{\futuresection}[1]{ \textit{#1}}
\newcommand{\futuresection}[1]{}

% The name of Apple's operating system
\newcommand{\osx}{OS~X\xspace}

% Proper typesetting of C++.
\newcommand{\cpp}{C\kern-0.15ex{+}\kern-0.1ex{+}\xspace}

\newcommand{\be}{\begin{enumerate}}
\newcommand{\ee}{\end{enumerate}}

\newcommand{\bi}{\begin{itemize}}
\newcommand{\ei}{\end{itemize}}

\newcommand{\art}{\emph{art}\xspace} %needs package xspace
\newcommand{\artdaq}{\emph{artdaq}\xspace} %needs package xspace
\newcommand{\fcl}{FHiCL\xspace} 
\newcommand{\larsoft}{LArSoft\xspace}

% Version numbers that must be kept consistent.
\newcommand{\docdbVersion}{395310-v1}  % Version number of this document
\newcommand{\gitTag}{v0.0.1}       % Version number of this document
%\newcommand{\cppTarVersion}{v0\_91} % Version number of the C++UpToSpeed.tar.gz file
                                    % Usually the version number of the first version of the
                                    % pdf file that requires this tar.gz file

%\newcommand{\gccv}{v4\_9\_3\xspace} % GCC version number for toyExperiment packages
%                                    % Remember to change this also in: GettingYourC++UpToSpeed/setup.sh 
%\newcommand{\toyv}{v0\_00\_30\xspace\xspace}  % git version number of the toyExperiment package
%\newcommand{\toyvClean}{v0\_00\_30}           % git version number of the toyExperiment package
%                                              %   - need version with no xspace to embed in urls.
%\newcommand{\wbv}{v0\_00\_39\xspace\xspace} % git version number of the art-workbook package
%                                                   %   - an accident if it is the same as the toyExperiment package!
%\newcommand{\downloadQuals}{s30-e9}           % Recommended qualifier set if you are downloading your own binaries.
%\newcommand{\setupQuals}{s30:e9}              % Same info needed in a different format in some places.
%\newcommand{\rootv}{v5.34/32\xspace}          % version of root used by the workbook
%\newcommand{\rooturl}{http://root.cern.ch/root/html534}  % url to version dependent root documentation.

\newcommand{\ddash}{-{}-} % Two consecutive dashes, not an endash.  Needed for option prefixes on unix commands.

% Keywords used by pullProducts for OSX releases.
% Check userguide/if-ug-installing-locally.tex for correctness
%
\newcommand{\mavericksKey}{d13}
\newcommand{\yosemiteKey}{d14}

\newcommand{\ingloss}[1]{\textit{#1}($\gamma$)\xspace}  

\newcommand{\cetbuildtools}{{\textbf{cetbuildtools}}\xspace} %needs package xspace
\newcommand{\fhiclcpp}{{\textbf{fhiclcpp}}\xspace} %needs package xspace

\newcommand{\fclstyle}{\color{blue}\begin{verbatim}} %AH adds 4/25/13  This works
\newcommand{\normstyle}{\end{verbatim}\color{black}} %AH adds 4/25/13  This doesn't work. Why not?

%\newcommand{\cmd}[1]{\texttt{\bf #1}}  %AH adds 6/12 per Brett 
\newcommand{\cmd}[1]{\colorbox{shadecolor}{\textsf{#1}}\xspace }  %% was   yellow!10
\newcommand{\cmdp}[1]{\colorbox{shadecolor}{\textsf{#1}}\xspace } % use for command inside a procedure box


\newcommand{\file}{\tt}  %AH adds 4/21/14 for filenames, file extensions and directory paths
\newcommand{\unix}{\sf}  %AH adds 4/21/14 for inline unix commands 
\newcommand{\utype}{\sf\textsl}  %AH adds for text user must substitute in commands
\newcommand{\code}{\tt}

% To typeset marginpar symbols for a left hand page, you need to
%  a) Right justify it within the marginpar box
%  b) In line numbers are turned on, then we need to not quite 
%     right justify, so that it leaves room for the line numbers.
%

% marginpar takes two args: \marginpar[left stuff]{right stuff} left is even, right is odd page.
\newcommand{\danger}{
\marginpar[{\hfill
\includegraphics[scale=0.25]{figures/KnuthDBend.png}\hspace{4 pt}
}]{
\includegraphics[scale=0.25]{figures/KnuthDBend.png}
}}  %AH adds 4/25/13  This works

\newcommand{\notabene}{
\marginpar[{\hfill % don't remove this; it affects invisible background   %\marginnote[{\hfill
\includegraphics[scale=0.5]{figures/pointc-right.png}   %\hspace{5 pt}
}]{
\includegraphics[scale=0.5]{figures/pointc-left.png}
}}  %AH adds 4/25/13  This works

\newcommand{\notice}{
\marginpar[{\hfill
\includegraphics[scale=0.5]{figures/notice.png}\hspace{8 pt}
}]{
\includegraphics[scale=0.5]{figures/notice.png}
}}  %AH adds 5/6/13  

\newcommand{\nonfermi}{
\marginpar[{\hfill
\includegraphics[scale=.3]{figures/ivy-univ.png}\hspace{8 pt}
}]{
\includegraphics[scale=.3]{figures/ivy-univ.png}
}}  %AH adds 5/8/13


\newcommand{\fermi}{
\marginpar[{\hfill
\includegraphics[scale=.3]{figures/wilsonflags.png}\hspace{8 pt}
}]{
\includegraphics[scale=.3]{figures/wilsonflags.png}
}}  %AH adds 5/8/13  

\newcommand{\logomutoe}{
\marginpar[{\hfill
\includegraphics[scale=.35]{figures/mu2e_logo_oval_100_58.png}\hspace{8 pt}
}]{
\includegraphics[scale=.35]{figures/mu2e_logo_oval_100_58.png}
}}  %AH adds 5/8/13  

\newcommand{\bomb}{
\marginpar[{\hfill
\includegraphics[scale=.4]{figures/bomb.png}\hspace{3 pt}
}]{
\includegraphics[scale=.4]{figures/bomb.png}
}}  %AH adds 5/14/13 

\newcommand{\gammapic}{
\marginpar[{\hfill
\includegraphics[scale=.4]{figures/gammapic.png}\hspace{3 pt}
}]{
\includegraphics[scale=.4]{figures/gammapic.png}
}}  %AH adds 4/4/14 


\def\Reco{{\tt Reco\ }}

                                             %%%%%%%   DEFAULT LISTSET   %%%%%%%
          
